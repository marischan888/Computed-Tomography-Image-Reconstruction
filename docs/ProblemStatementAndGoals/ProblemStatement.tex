\documentclass{article}

\usepackage{tabularx}
\usepackage{booktabs}

\title{Problem Statement and Goals\\\progname}

\author{\authname}

\date{\today}

%% Comments

\usepackage{color}

\newif\ifcomments\commentstrue %displays comments
%\newif\ifcomments\commentsfalse %so that comments do not display

\ifcomments
\newcommand{\authornote}[3]{\textcolor{#1}{[#3 ---#2]}}
\newcommand{\todo}[1]{\textcolor{red}{[TODO: #1]}}
\else
\newcommand{\authornote}[3]{}
\newcommand{\todo}[1]{}
\fi

\newcommand{\wss}[1]{\authornote{blue}{SS}{#1}} 
\newcommand{\plt}[1]{\authornote{magenta}{TPLT}{#1}} %For explanation of the template
\newcommand{\an}[1]{\authornote{cyan}{Author}{#1}}

%% Common Parts

\newcommand{\progname}{ProgName} % PUT YOUR PROGRAM NAME HERE
\newcommand{\authname}{Team \#, Team Name
\\ Student 1 name
\\ Student 2 name
\\ Student 3 name
\\ Student 4 name} % AUTHOR NAMES                  

\usepackage{hyperref}
    \hypersetup{colorlinks=true, linkcolor=blue, citecolor=blue, filecolor=blue,
                urlcolor=blue, unicode=false}
    \urlstyle{same}
                                


\begin{document}

\maketitle

\begin{table}[hp]
\caption{Revision History} \label{TblRevisionHistory}
\begin{tabularx}{\textwidth}{llX}
\toprule
\textbf{Date} & \textbf{Developer(s)} & \textbf{Change}\\
\midrule
  January 16, 2025 & Qianlin Chen & Initial Draft \\
\midrule
  ... & ... & ...\\
\bottomrule
\end{tabularx}
\end{table}

\section{Problem Statement}

\subsection{Problem}
% Background
With the arrival of Computed Tomography (CT) as a diagnostic tool in medical
imaging, X-ray imaging underwent a revolution. Tomography is a method of imaging
a two- or three-dimensional object from multiple one-dimensional ``slices'' of
the object. In a CT scan, these slices are created by multiple parallel X-ray
beams passing through the object at varying angles. The initial and final
intensity of each beam is recorded, and the original image is reconstructed
using backprojection with data from multiple slices.
%% What is the problem and Why is this problem matter:
%% TODO: citation of problem
\newline However, significant noise blurs the recreated image, even as the number of
backprojections increases. Regardless of the number of directions used for
backprojection, it cannot perfectly recreate the image using the backprojection
formula. Therefore, it is necessary to develop techniques to filter out noise
created by backprojection and produce a smoother representation of the object.
Additionally, different filtering techniques may yield varying reconstruction
efficiencies, so selecting an appropriate filter is crucial.

\subsection{Inputs and Outputs}
\subsubsection{Inputs}
\begin{itemize}
\item Phantom images.
\item Sinogram data.
\item Projection angles in degrees.
\item Filter type to be applied during backprojection.
\end{itemize}

\subsubsection{Outputs}
\begin{itemize}
  \item Reconstructed CT images.
\end{itemize}

\subsection{Stakeholders}
\begin{itemize}
\item Researchers
\item Hospital
\end{itemize}

\subsection{Environment}
\begin{description}
\item[Software] Windows, Linux or Mac OS
\end{description}

\section{Goals}
\begin{description}
\item[Image Quality Enhancement] \hfill \\ The quality of reconstructed images will
  improve through advanced filtering techniques.
\item[Filter Options] \hfill \\ High-pass and low-pass filters will be available to
  enhance flexibility and optimize reconstruction.
\item[User-Friendly Design] \hfill \newline The application will be intuitive, requiring
  no additional instructions for users to understand all features.
\end{description}

\section{Stretch Goals}
\begin{description}
\item[Adaptive Filtering] \hfill \\ Implement adaptive filters that automatically
  adjust based on the image characteristics, reducing manual intervention.
\item[Real-Time Reconstruction] \hfill \\ Develop functionality for real-time
  image reconstruction during the scanning process, enabling faster
  diagnostics.
\end{description}

\section{Challenge Level and Extras}
The primary challenge of this project lies in the integration of domain
knowledge from the medical and mathematical fields. Understanding the principles
of medical imaging, including tomography and the Radon transform, as well as
mastering the associated mathematical concepts, requires significant learning
and effort.\\
While coding is not a challenge, testing poses another difficulty
due to limited access to high-quality data, which could impact the evaluation of
reconstruction accuracy and filter performance. Addressing these issues will
demand innovative approaches, such as using simulated data or augmenting limited
datasets.

\wss{State your expected challenge level (advanced, general or basic).  The
challenge can come through the required domain knowledge, the implementation or
something else.  Usually the greater the novelty of a project the greater its
challenge level.  You should include your rationale for the selected level.
Approval of the level will be part of the discussion with the instructor for
approving the project.  The challenge level, with the approval (or request) of
the instructor, can be modified over the course of the term.}

\wss{Teams may wish to include extras as either potential bonus grades, or to
make up for a less advanced challenge level.  Potential extras include usability
testing, code walkthroughs, user documentation, formal proof, GenderMag
personas, Design Thinking, etc.  Normally the maximum number of extras will be
two.  Approval of the extras will be part of the discussion with the instructor
for approving the project.  The extras, with the approval (or request) of the
instructor, can be modified over the course of the term.}

\newpage{}

\section*{Appendix --- Reflection}

\wss{Not required for CAS 741}

The purpose of reflection questions is to give you a chance to assess your own
learning and that of your group as a whole, and to find ways to improve in the
future. Reflection is an important part of the learning process.  Reflection is
also an essential component of a successful software development process.  

Reflections are most interesting and useful when they're honest, even if the
stories they tell are imperfect. You will be marked based on your depth of
thought and analysis, and not based on the content of the reflections
themselves. Thus, for full marks we encourage you to answer openly and honestly
and to avoid simply writing ``what you think the evaluator wants to hear.''

Please answer the following questions.  Some questions can be answered on the
team level, but where appropriate, each team member should write their own
response:


\begin{enumerate}
    \item What went well while writing this deliverable?
    \item What pain points did you experience during this deliverable, and how
    did you resolve them?
    \item How did you and your team adjust the scope of your goals to ensure
    they are suitable for a Capstone project (not overly ambitious but also of
    appropriate complexity for a senior design project)?
\end{enumerate}

\end{document}
