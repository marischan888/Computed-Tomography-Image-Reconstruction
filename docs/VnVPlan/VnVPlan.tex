\documentclass[12pt, titlepage]{article}

\usepackage{booktabs}
\usepackage{tabularx}
\usepackage{hyperref}
\hypersetup{
    colorlinks,
    citecolor=blue,
    filecolor=black,
    linkcolor=red,
    urlcolor=blue
}
\usepackage[round]{natbib}
\usepackage{float}
\restylefloat{table}
\usepackage{tikz}

%% Comments

\usepackage{color}

\newif\ifcomments\commentstrue %displays comments
%\newif\ifcomments\commentsfalse %so that comments do not display

\ifcomments
\newcommand{\authornote}[3]{\textcolor{#1}{[#3 ---#2]}}
\newcommand{\todo}[1]{\textcolor{red}{[TODO: #1]}}
\else
\newcommand{\authornote}[3]{}
\newcommand{\todo}[1]{}
\fi

\newcommand{\wss}[1]{\authornote{blue}{SS}{#1}} 
\newcommand{\plt}[1]{\authornote{magenta}{TPLT}{#1}} %For explanation of the template
\newcommand{\an}[1]{\authornote{cyan}{Author}{#1}}

%% Common Parts

\newcommand{\progname}{ProgName} % PUT YOUR PROGRAM NAME HERE
\newcommand{\authname}{Team \#, Team Name
\\ Student 1 name
\\ Student 2 name
\\ Student 3 name
\\ Student 4 name} % AUTHOR NAMES                  

\usepackage{hyperref}
    \hypersetup{colorlinks=true, linkcolor=blue, citecolor=blue, filecolor=blue,
                urlcolor=blue, unicode=false}
    \urlstyle{same}
                                


\newcounter{testnum}
\newcommand{\dthetestnum}{id\thetestnum}
\newcommand{\tref}[1]{id\ref{#1}}

\begin{document}

\title{System Verification and Validation Plan for \progname{}}
\author{\authname}
\date{\today}
\maketitle

\pagenumbering{roman}

\section*{Revision History}

\begin{tabularx}{\textwidth}{p{3cm}p{2cm}X}
\toprule {\bf Date} & {\bf Version} & {\bf Notes}\\
\midrule
Feb 24, 2025 & 1.0 & Initial VnV\\
... & ... & ...\\
\bottomrule
\end{tabularx}

~\\

\newpage

\tableofcontents

\listoftables

%\listoffigures

\newpage

\section{Symbols, Abbreviations, and Acronyms}

\renewcommand{\arraystretch}{1.2}
\begin{tabular}{l l}
  \toprule
  \textbf{symbol} & \textbf{description}\\
  \midrule
  MG & Module Guide \\
  MIS & Module Interface Specification \\
  SRS & Software Requirement Specification\\
  FBP & Filter Back Projection \\
  TC & Test Case \\
  VnV & Verification and Validation \\
  \bottomrule
\end{tabular}\\

For complete symbols used within the system, please refer the section 1 in
  \href{https://github.com/marischan888/Computed-Tomography-Image-Reconstruction/blob/main/docs/SRS/SRS.pdf}{SRS}
  document.

\newpage

\pagenumbering{arabic}

This document outlines the VnV plan for FBP CT Image Reconstruction, ensuring compliance with
functional and non-functional requirements through systematic verification and
testing. It begins with General Information (Section~\ref{sec2}), covering objectives and
relevant documentation, followed by the VnV Plan (Section~\ref{sec3}), which details team
responsibilities, SRS and design verification, and automated testing tools.
System Tests (Section~\ref{sec4}) validate functional and non-functional requirements,
including input and output tests, with traceability ensuring full requirement
coverage. Lastly, Unit Test Description (Section~\ref{sec5}) defines the testing scope,
module-specific functional and non-functional tests, and the mapping of test
cases to system components.

\section{General Information} \label{sec2}

\subsection{Summary}
This document reviews the verification and validation plan for Filtered Back
Projection (FBP) CT Image Reconstruction. The FBP method is used to reconstruct
cross-sectional images from projection data collected in CT scans. The accuracy
and performance of the reconstruction process are evaluated based on predefined
validation criteria. The system’s usability, maintainability, and reliability
are also assessed using user inputs and experimental results.

\subsection{Objectives}
The purpose of the validation plan is to define how system validation will
perform at the end of the project. The strategy will use to assess whether the
developed system accomplishes the design goals. Also, the verification plan
includes test strategies, definitions of what will be tested, and a test matrix
with detailed mapping connecting the tests performed to the system
requirements.

Due to resource constraints, validation will be limited to
peer-based testing for accuracy and usability. While this ensures basic
functionality and user experience evaluation, the absence of domain experts,
such as doctors or researchers, prevents validation in real-world clinical or
research applications.

\subsection{Challenge Level and Extras}
The challenge level for this project is classified as general, requiring a
rigorous verification and validation process to ensure the accuracy and
performance of the system. The validation plan includes extensive functional and non-functional
testing, leveraging automated tools, formal proofs, and usability evaluations to
assess correctness, efficiency, and maintainability.

Additionally, this project incorporates code walkthroughs, user documentation,
and design reviews to enhance system reliability and comprehensibility. The
scope of validation covers core algorithmic accuracy while excluding external
dependencies, which are assumed to be pre-validated. This structured approach
ensures that the system meets its defined performance, usability, and
verification requirements while adhering to resource constraints.


\subsection{Relevant Documentation}
The relevant documentation for the VnV includes
\href{https://github.com/marischan888/Computed-Tomography-Image-Reconstruction/blob/main/docs/ProblemStatementAndGoals/ProblemStatement.pdf}{Problem
  Statement} which identifies the proposed idea,
\href{https://github.com/marischan888/Computed-Tomography-Image-Reconstruction/blob/main/docs/SRS/SRS.pdf}{System
  Requirements Specifications} which provides information about the requirement
of the proposed system, VnV Report for validation and verification, MG and MIS
design documents for unit testing (found in
\href{https://github.com/marischan888/Computed-Tomography-Image-Reconstruction/tree/main/docs/Design}{Github
  Repository}).

\section{Plan} \label{sec3}
This section describes the verification and validation plan for the
\progname.The planning starts with the verification and validation team in
Section \ref{3.1}, followed by the SRS verification plan (Section \ref{3.2}),
design verification plan (Section \ref{3.3}), verification and validation
verification plan (Section \ref{3.4}), implementation verification plan (Section
\ref{3.5}), Automated testing and verification tools (Section \ref{3.6}), and
Software validation plan (Section \ref{3.7}).

\subsection{Verification and Validation Team} \label{3.1}
\begin{center}
\begin{table}[h]
    \begin{tabular}{ |l|l|p{2cm}|p{5cm}| }
    \hline
    \textbf{Name} & \textbf{Document} & \textbf{Role} & \textbf{Description} \\
    \hline
     Dr. Spencer Smith & All & Instructor/ Reviewer & Review the documents, design and documentation style. \\
     \hline
     Qianlin Chen & All & Author & Create and manage all documentation, develop the VnV plan, conduct VnV testing, and verify the implementation.\\
     \hline
     Xunzhou Ye & All & Domain Expert Reviewer & Review all the documents. \\
    \hline
    \end{tabular} %
\caption{Verification and validation team}
\label{vnvteam}
\end{table}
\end{center}

\subsection{SRS Verification Plan} \label{3.2}
The SRS document for \progname will be verified using a structured review
process. The verification plan consists of the following steps:
\begin{description}
\item[Initial Review] \hfill \\
  Assigned reviewers, instructor (Dr. Smith) and domain expert
  (Xunzhou). will conduct a manual review of the SRS document. In this step, a
  structured
  \href{https://github.com/smiths/capTemplate/blob/9251702fdcb9800c59f6ed3d11d91e2bd62fca6d/docs/Checklists/SRS-Checklist.pdf}{SRS
    Checklist} will be used to systematically evaluate key aspects such as
  requirement completeness, consistency, feasibility, and traceability.
\item[Issue Tracking] \hfill \\
  eviewers will provide feedback by documenting identified issues in an issue
  tracker (GitHub Issues). Each issue will be
  assigned to the document owner for later revision.
\item[Revision] \hfill \\
  The document owner will address reported issues by making necessary
  modifications to the SRS. Any unresolved or disputed concerns will be
  discussed among the reviewers.
\item[Final Documentation] \hfill \\
  After revisions, a final review will be conducted to confirm that all
  identified issues have been adequately addressed.
\end{description}


\subsection{Design Verification Plan} \label{3.3} The design documentation,
including the Module Guide (MG) and Module Interface Specification (MIS), will
be reviewed to ensure accuracy and completeness. The verification process will
involve a static analysis approach, where assigned reviewers will inspect the
documents to confirm that they align with system requirements and architectural
design principles. Reviewers will provide feedback through an issue-tracking
system, allowing the document owner to address concerns systematically. \\
A structured
\href{https://github.com/smiths/capTemplate/blob/9251702fdcb9800c59f6ed3d11d91e2bd62fca6d/docs/Checklists/MG-Checklist.pdf}{MG
  Checklist} and
\href{https://github.com/smiths/capTemplate/blob/9251702fdcb9800c59f6ed3d11d91e2bd62fca6d/docs/Checklists/MIS-Checklist.pdf}{MIS
  Checklist} will be used to ensure consistency, correctness, and traceability.
The finalized documents will be reviewed again to validate that all identified
issues have been resolved.

\subsection{Verification and Validation Plan Verification Plan} \label{3.4} The
review process for Verification and Validation (VnV) Plan will involve assigned
team members in Table \ref{vnvteam} conducting a structured inspection to verify that the plan aligns
with project requirements and verification strategies.
Identified issues will be documented in an issue-tracking system for resolution.\\
A
\href{https://github.com/smiths/capTemplate/blob/9251702fdcb9800c59f6ed3d11d91e2bd62fca6d/docs/Checklists/VnV-Checklist.pdf}{VnV
  Checklist} will be used to assess key aspects of the plan, including coverage
of testing methodologies and traceability of requirements. The final review will
confirm that all necessary corrections have been implemented before approval.

\subsection{Implementation Verification Plan} \label{3.5}

\begin{itemize}
\item Static Verification:
  \begin{itemize}
      \item Code Walkthrough: A structured review process where team members
        manually inspect the code for correctness, maintainability, and
        adherence to design specifications. This process ensures that critical
        components meet functional requirements before execution.
  \end{itemize}
  \item Dynamic Verification:
    \begin{itemize}
      \item Unit Testing: All test cases outlined in Section~\ref{sec5} will
        be executed, covering both functional and non-functional requirements.
        These tests ensure correctness at the module and system levels,
        validating reconstruction accuracy and computational efficiency.
      \end{itemize}
  \end{itemize}

\subsection{Automated Testing and Verification Tools} \label{3.6} To ensure the
correctness and efficiency of the \progname, the following automated
testing and verification tools will be utilized:
\begin{description}
\item[Unit Testing] \hfill \\
  \href{https://docs.pytest.org/en/stable/}{pytest} will be used to test
  individual functions, such as backprojection and filtering operations, to
  verify that they produce the expected output given test input data.
\item[Performance Analysis] \hfill \\
  \href{https://docs.python.org/3/library/profile.html#module-cprofile}{cProfill}
  will analyze execution time and identify performance bottlenecks in
  computationally intensive functions like fourier transforms and interpolation.
\item[Static Analysis] \hfill \\
  \href{https://flake8.pycqa.org/en/latest/}{flake8} will enforce python code
  standards, ensuring maintainability and readability.
\item[Test Coverage] \hfill \\
  \href{https://coverage.readthedocs.io/en/7.6.12/}{coverage.py} will measure
  test coverage to ensure that all critical functions are tested and validated
  against various datasets.
\item[Visualization] \hfill \\
  \href{https://matplotlib.org/}{matplotlib} will be used to visualize sinograms
  and reconstructed images, allowing for manual verification of reconstruction
  accuracy.
\end{description}
Continuous Integration (CI) tools will be Github Action. Tools for
\href{https://github.com/marketplace/actions/run-pytest}{unit testing},
\href{https://github.com/marketplace/actions/flake8-action}{static analysis} can
run automatically.

\subsection{Software Validation Plan} \label{3.7} Currently, there are no
external datasets or domain experts available to validate the FBP CT Image
Reconstruction system. Since the system is not yet in real-world use and lacks
access to researcher for review, formal validation cannot be conducted at this
stage.

However, internal review sessions may be used to ensure that the system
aligns with specified requirements. Future validation efforts could include
structured user testing or a Rev 0 demo if external stakeholders become
available. Until then, verification efforts will focus on functional correctness
through testing, as outlined in the SRS verification section.


\section{System Tests} \label{sec4} This section outlines the testing framework
for verifying the system, covering both functional and non-functional
requirements. Section~\ref{TFR} details functional tests, including input tests
(\ref{output_test}) to validate data preprocessing and output tests
(\ref{input_test}) to assess reconstructed image accuracy. Section~\ref{TNFR}
focuses on non-functional testing, such as performance and usability
evaluations. Finally, Section~\ref{trace1} ensures traceability between test cases and
system requirements, maintaining comprehensive coverage and validation of the
system’s expected behavior.

\subsection{Tests for Functional Requirements} \label{TFR}
The functional requirements in \href{https://github.com/marischan888/Computed-Tomography-Image-Reconstruction/blob/main/docs/SRS/SRS.pdf}{SRS} are divided into input and output
tests. \\
R1 and R2 correspond to input-related tests, ensuring that the system
correctly processes input images, projection data, and transformations. The
tests for R1 and R2 are in the Section \ref{input_test} \\
R3 correspond to output-related tests, verifying the correctness of
Fourier filtering, back-projection, and reconstructed attenuation values. The
test for the R3 is in the Section \ref{output_test} \\

\subsubsection{Input Tests} \label{input_test}

\paragraph{Test for Input - 2D grayscale image}
% Add reference for the input
\begin{enumerate}
\item{test-input-image-\refstepcounter{testnum} \label{id1}\dthetestnum}
\begin{description}
\item[Control:] Automatic

\item[Initial State:] Pending Input.

\item[Input:] \hfill \\
  \begin{itemize}
  \item A 2-D array represents the intensity measurements in different projection
    angles. Sample data input can be found in
    \href{https://www.cancerimagingarchive.net/collection/ldct-and-projection-data/}{Cancer
      Image Archive} under the title \lq Images Phantom Object Only\rq.
  \item  The projection angles in degrees. In this test, a range of 180 angles from 0 to
    179 degrees is used.
  \end{itemize}

\item[Output:] A 2-D array represents the Sinogram. The result for the input data
  can be found in
  \href{https://www.cancerimagingarchive.net/collection/ldct-and-projection-data/}{Cancer
    Image Archive} under the title \lq Clinical Data\rq.

\item[Test Case Derivation:] This test ensures that the system correctly accepts a
  valid grayscale 2D input image, applies the Radon Transform, and produces a
  valid sinogram matrix. \\
  The system should return an output that maintains consistency in shape and
  correctly reflects the values after applying the Radon Transform.

\item[How test will be performed:] The test will be automated using PyTest.
\end{description}
\end{enumerate}

\paragraph{Test for Input - filter type}
% Add a table, given all input and their expected output

\begin{enumerate}
\item{test-input-filter-type-\refstepcounter{testnum} \label{id2} \dthetestnum}
\begin{description}
\item[Control:] Automatic

\item[Initial State:] System has generated the sinogram matrix and the user is
  ready to select the ramp filter.

\item[Input:] Shepp-logan filter type and the size of sinogram in the Table\ref{TableA}.

\item[Output:] A 1-D array that represent the selected Fourier filter in the Table\ref{TableA}.

\item[Test Case Derivation:] \hfill \\
  This test ensures that the system correctly processes
  user-selected filter types.\\
  A valid ramp filter should be applied as expected, modifying the sinogram
  by filtering out the low frequency.

\item[How test will be performed:] The test will be automated using PyTest.
\end{description}
\end{enumerate}

\begin{enumerate}
\item{test-input-filter-type-\refstepcounter{testnum} \label{id3} \dthetestnum}
\begin{description}
\item[Control:] Automatic


\item[Initial State:] System has generated the sinogram matrix and the user is
  ready to select the ramp filter.

\item[Input:] Shepp-logan filter type and the size of sinogram in the Table\ref{TableA}.

\item[Output:] A 1-D array that represent the selected Fourier filter in the Table\ref{TableA}.

\item[Test Case Derivation:] \hfill \\
  This test ensures that the system correctly processes
  user-selected filter types.\\
  A valid shepp-logan filter should be applied as expected, modifying the
  sinogram by filtering out the high frequency.

\item[How test will be performed:] The test will be automated using PyTest.
\end{description}
\end{enumerate}

\begin{enumerate}
\item{test-input-invalid-filter-type-\refstepcounter{testnum} \label{id4} \dthetestnum}
\begin{description}
\item[Control:] Automatic


\item[Initial State:] System has generated the sinogram matrix and the user is
  ready to select the ramp filter.

\item[Input:] No filter type and the size of sinogram in the Table\ref{TableA}.

\item[Output:] A 1-D array that represent the selected Fourier filter in the Table\ref{TableA}.

\item[Test Case Derivation:] \hfill \\
  This test ensures that the system correctly processes
  user-selected filter types.\\
  A valid shepp-logan filter should be applied as expected, modifying the
  sinogram by filtering out the high frequency.

\item[How test will be performed:] The test will be automated using PyTest.
\end{description}
\end{enumerate}

\begin{table}[H]
  \centering
  \begin{tabularx}{\linewidth}{|X|X|X|}
    \hline
    Test Id                                  & Input                                            & Expected Output                                                                                                                             \\
    \hline
    test-input-filter-type-id\ref{id2}         & Filter Type: ramp (str); Sinogram size: 10 (int)  & 1-D array represent ramp filter variable: [0.0, 0.314159, 0.628318, 0.942477, 1.256636, 1.570796, 1.256636, 0.942477, 0.628318, 0.314159] \\ \hline
    test-input-filter-type-id\ref{id3}         & Filter Type: shepp (str); Sinogram size: 10 (int) & 1-D array represent shepp-logan filter variable: [0.0, 0.2984, 0.5630, 0.7958, 0.9817, 1.0986, 0.9817, 0.7958, 0.5630, 0.2984]            \\ \hline
    test-input-invalid-filter-type-id\ref{id4} & anything other than ramp and shepp               & None                                                                                                                                        \\ \hline
  \end{tabularx}
  \caption{The inputs and outputs for filter tests}
  \label{TableA}
\end{table}


\subsubsection{Output Tests}\label{output_test}
\paragraph{Test for Output - attenuation}
\begin{enumerate}
\item{test-output-attenuation-\refstepcounter{testnum} \label{id5} \dthetestnum}
\begin{description}
\item[Control:] Automatic

\item[Initial State:] The system is ready to perform back-projection on a given
  sinogram and filter type.

\item[Input:] A sinogram and the ramp filter in the Table \ref{Table B}.

\item[Output:] A 2-D array that represent the reconstructed attenuation in the
  Table\ref{Table B}.

\item[Test Case Derivation:] This test ensures that the system correctly applies the back-projection process to reconstruct an image from a given sinogram. The test verifies that the output maintains the expected shape and that the reconstructed attenuation values correctly represent the original structure.

\item[How test will be performed:] The test will be automated using PyTest.
\end{description}
\end{enumerate}

\begin{enumerate}
\item{test-output-attenuation-\refstepcounter{testnum} \label{id6} \dthetestnum}
\begin{description}
\item[Control:] Automatic

\item[Initial State:] The system is ready to perform back-projection on a given
  sinogram and filter type.

\item[Input:] A sinogram and the shepp-logan filter in the Table \ref{Table B}.

\item[Output:] A 2-D array that represent the reconstructed attenuation in the
  Table\ref{Table B}.

\item[Test Case Derivation:] This test ensures that the system correctly applies the back-projection process to reconstruct an image from a given sinogram. The test verifies that the output maintains the expected shape and that the reconstructed attenuation values correctly represent the original structure.

\item[How test will be performed:] The test will be automated using PyTest.
\end{description}
\end{enumerate}


\begin{table}[H]
  \centering
  \begin{tabularx}{\linewidth}{|X|X|X|}
    \hline
    Test Id                                  & Input                                            & Expected Output                                                                                                                             \\
    \hline
    test-output-attenuation-id\ref{id5}         & Sinogram can be found in
                                                   \href{https://www.cancerimagingarchive.net/collection/ldct-and-projection-data/}{Cancer
                                                   Image Archive} under the
                                                   title \lq Clinical
                                                   Data\rq; Filter Type:
                                                   ramp (str); Output size:
                                                   256 & Attenuation can be found in
                                                   \href{https://www.cancerimagingarchive.net/collection/ldct-and-projection-data/}{Cancer
                                                   Image Archive} under the
                                                   title \lq Patients Report\rq \\ \hline
    test-output-attenuation-id\ref{id6}         & Sinogram can be found in
                                                       \href{https://www.cancerimagingarchive.net/collection/ldct-and-projection-data/}{Cancer
                                                       Image Archive} under the
                                                       title \lq Clinical
                                                       Data\rq; Filter Type:
                                                       shepp (str); Output size:
                                                       256 & Attenuation can be found in
                                                       \href{https://www.cancerimagingarchive.net/collection/ldct-and-projection-data/}{Cancer
                                                       Image Archive} under the
                                                       title \lq Patients
                                                             Report\rq \\ \hline
  \end{tabularx}
  \caption{The inputs and outputs for FBP tests}
  \label{Table B}
\end{table}
\subsection{Tests for Nonfunctional Requirements} \label{TNFR}
Nonfunctional requirements for FBP system are given in
\href{https://github.com/marischan888/Computed-Tomography-Image-Reconstruction/blob/main/docs/SRS/SRS.pdf}{SRS}
Section 5.2.

\paragraph{Test for Accuracy}

\begin{enumerate}

\item{test-accuracy-\refstepcounter{testnum} \label{id7} \dthetestnum\\}

Type: Functional, Dynamic.

Initial State: Reconstructed image is available.

Input/Condition: The reconstructed image and the corresponding ground truth image.

Output/Result: Error margin (Mean Square Error) is computed and compared against the predefined tolerance level.

How test will be performed: The test will compare the reconstructed images with
ground-truth images using metrics like Mean Squared Error (MSE). The relative
error for each test case will be recorded and analyzed.

\end{enumerate}

\paragraph{Test for Usability}

\begin{enumerate}

\item{test-usability-\refstepcounter{testnum} \label{id8} \dthetestnum\\}

Type: Manual.

Initial State: System interface is implemented. Usability survey in Section
\ref{survey} is predefined.

Input/Condition: Joe in Table \ref{vnvteam} perform tasks follows the survey.

Output/Result: Feedback from usability survey, including ease of use, clarity of
instructions, and efficiency of interaction.

How test will be performed: A usability study will be conducted where users
perform predefined tasks. Their experience will be recorded, and a survey will
be provided to assess ease of use, with results summarized in a report.
\end{enumerate}


\paragraph{Test for Maintainability}

\begin{enumerate}

\item{test-maintainability-\refstepcounter{testnum} \label{id9} \dthetestnum\\}

Type: Manual, Static.

Initial State: System implementation is available.

How test will be performed:\\
To ensure the maintainability of the system, a structured code review process
will be conducted. The review will focus on the clarity of documentation, ease of
understanding the code structure, and the effort required for future
modifications or enhancements.

Additionally, maintainability will be assessed by tracking modification times
for common updates, such as adding new filters or adjusting reconstruction
parameters. If changes require excessive refactoring, recommendations for
improving the code's modularity will be documented.

The review will be conducted collaboratively, involving the assigned document
reviewers listed in Table \ref{vnvteam}. The findings will be summarized in a
Github Issue, outlining strengths and areas for improvement.
\end{enumerate}

\paragraph{Test for Portability}

\begin{enumerate}

\item{test-portability-\refstepcounter{testnum} \label{id10} \dthetestnum\\}

Type: Functional, Dynamic.

Initial State: The software is fully implemented.

Input/Condition: Different operating systems (Windows, macOS, Linux).

Output/Result: Verification that the system runs without errors across all tested platforms.

How test will be performed: The software will be deployed and executed on
multiple operating systems, ensuring it performs consistently without requiring
additional setup.
\end{enumerate}


\paragraph{Test for Reusability}

\begin{enumerate}

\item{test-reusability-\refstepcounter{testnum} \label{id11} \dthetestnum\\}

Type: Manual.

Initial State: API is implemented and documented.

Input/Condition: External developers in Table \ref{vnvteam} integrating a new filter beyond Ramp and Shepp-Logan.

Output/Result: Confirmation that new filters can be added with minimal effort using the provided API.

How test will be performed: Developers will attempt to add a new filter, and the
time/effort required will be assessed. Documentation clarity and API flexibility
will also be evaluated.
\end{enumerate}

\subsection{Traceability Between Test Cases and Requirements} \label{trace1}
\begin{table}[h!]
\centering
\begin{tabular}{|c|c|c|c|c|c|c|c|c|}
\hline
  \tikz{\node[below left, inner sep=1pt] (test) {test};%
      \node[above right,inner sep=1pt] (requirement) {requirement};%
      \draw (test.north west|-requirement.north west) -- (test.south east-|requirement.south east);}
            & R1 & R2 & R3 & NFR1 & NFR2 & NFR3 & NFR4 & NFR5 \\
\hline
\tref{id1}  & X  &    &    &      &      &      &      &      \\ \hline
\tref{id2}  &    & X  &    &      &      &      &      &      \\ \hline
\tref{id3}  &    & X  &    &      &      &      &      &      \\ \hline
\tref{id4}  &    & X  &    &      &      &      &      &      \\ \hline
\tref{id5}  &    &    & X  &      &      &      &      &      \\ \hline
\tref{id6}  &    &    & X  &      &      &      &      &      \\ \hline
\tref{id7}  &    &    &    & X    &      &      &      &      \\ \hline
\tref{id8}  &    &    &    &      & X    &      &      &      \\ \hline
\tref{id9}  &    &    &    &      &      & X    &      &      \\ \hline
\tref{id10} &    &    &    &      &      &      & X    &      \\ \hline
\tref{id11} &    &    &    &      &      &      &      & X    \\ \hline
\end{tabular}
\caption{Traceability Matrix Showing the Connections Between Requirements and
  Test Cases}
\label{Table:R_trace}
\end{table}

\section{Unit Test Description} \label{sec5}

\wss{This section should not be filled in until after the MIS (detailed design
  document) has been completed.}

\wss{Reference your MIS (detailed design document) and explain your overall
philosophy for test case selection.}

\wss{To save space and time, it may be an option to provide less detail in this section.
For the unit tests you can potentially layout your testing strategy here.  That is, you
can explain how tests will be selected for each module.  For instance, your test building
approach could be test cases for each access program, including one test for normal behaviour
and as many tests as needed for edge cases.  Rather than create the details of the input
and output here, you could point to the unit testing code.  For this to work, you code
needs to be well-documented, with meaningful names for all of the tests.}

\subsection{Unit Testing Scope}

\wss{What modules are outside of the scope.  If there are modules that are
  developed by someone else, then you would say here if you aren't planning on
  verifying them.  There may also be modules that are part of your software, but
  have a lower priority for verification than others.  If this is the case,
  explain your rationale for the ranking of module importance.}

\subsection{Tests for Functional Requirements}

\wss{Most of the verification will be through automated unit testing.  If
  appropriate specific modules can be verified by a non-testing based
  technique.  That can also be documented in this section.}

\subsubsection{Module 1}

\wss{Include a blurb here to explain why the subsections below cover the module.
  References to the MIS would be good.  You will want tests from a black box
  perspective and from a white box perspective.  Explain to the reader how the
  tests were selected.}

\begin{enumerate}

\item{test-id1\\}

Type: \wss{Functional, Dynamic, Manual, Automatic, Static etc. Most will
  be automatic}

Initial State:

Input:

Output: \wss{The expected result for the given inputs}

Test Case Derivation: \wss{Justify the expected value given in the Output field}

How test will be performed:

\item{test-id2\\}

Type: \wss{Functional, Dynamic, Manual, Automatic, Static etc. Most will
  be automatic}

Initial State:

Input:

Output: \wss{The expected result for the given inputs}

Test Case Derivation: \wss{Justify the expected value given in the Output field}

How test will be performed:

\item{...\\}

\end{enumerate}

\subsubsection{Module 2}

...

\subsection{Tests for Nonfunctional Requirements}

\wss{If there is a module that needs to be independently assessed for
  performance, those test cases can go here.  In some projects, planning for
  nonfunctional tests of units will not be that relevant.}

\wss{These tests may involve collecting performance data from previously
  mentioned functional tests.}

\subsubsection{Module ?}

\begin{enumerate}

\item{test-id1\\}

Type: \wss{Functional, Dynamic, Manual, Automatic, Static etc. Most will
  be automatic}

Initial State:

Input/Condition:

Output/Result:

How test will be performed:

\item{test-id2\\}

Type: Functional, Dynamic, Manual, Static etc.

Initial State:

Input:

Output:

How test will be performed:

\end{enumerate}

\subsubsection{Module ?}

...

\subsection{Traceability Between Test Cases and Modules}

\wss{Provide evidence that all of the modules have been considered.}

%\bibliographystyle{plainnat}
%
%\bibliography{../../refs/References}

\newpage

\section{Appendix}

This is where you can place additional information.

\subsection{Symbolic Parameters}

The definition of the test cases will call for SYMBOLIC\_CONSTANTS.
Their values are defined in this section for easy maintenance.

\subsection{Sample Usability Survey Questions}\label{survey}
Thank you for participating in this usability survey. Your feedback will help us improve the system’s usability and user experience.

\textbf{Participant Information:}
\begin{itemize}
    \item Role: \_\_\_\_\_\_\_\_\_\_\_ (e.g., Researcher, Student, Developer)
    \item Experience with similar tools: \_\_\_\_\_\_\_\_\_\_\_ (None / Beginner / Intermediate / Expert)
    \item Operating system used: \_\_\_\_\_\_\_\_\_\_\_ (Windows / macOS / Linux)
\end{itemize}

\textbf{Task Performance:}
On a scale from 1 (very difficult) to 5 (very easy), rate how easy it was to complete the following tasks:
\begin{center}
    \begin{tabular}{|l|c|c|c|c|c|}
        \hline
        \textbf{Task} & 1 & 2 & 3 & 4 & 5 \\
        \hline
        Loading projection data & ☐ & ☐ & ☐ & ☐ & ☐ \\
        Selecting a filter & ☐ & ☐ & ☐ & ☐ & ☐ \\
        Generating a sinogram & ☐ & ☐ & ☐ & ☐ & ☐ \\
        Applying back-projection & ☐ & ☐ & ☐ & ☐ & ☐ \\
        Viewing and comparing images & ☐ & ☐ & ☐ & ☐ & ☐ \\
        \hline
    \end{tabular}
\end{center}

\textbf{System Usability:}
Please rate the following aspects of the system on a scale from 1 (strongly disagree) to 5 (strongly agree).
\begin{center}
    \begin{tabular}{|l|c|c|c|c|c|}
        \hline
        \textbf{Statement} & 1 & 2 & 3 & 4 & 5 \\
        \hline
        The system was easy to use. & ☐ & ☐ & ☐ & ☐ & ☐ \\
        The instructions were clear. & ☐ & ☐ & ☐ & ☐ & ☐ \\
        I was able to complete tasks efficiently. & ☐ & ☐ & ☐ & ☐ & ☐ \\
        The system's response time was acceptable. & ☐ & ☐ & ☐ & ☐ & ☐ \\
        I would use this system again. & ☐ & ☐ & ☐ & ☐ & ☐ \\
        \hline
    \end{tabular}
\end{center}

\textbf{Open-Ended Feedback:}
\begin{itemize}
    \item What did you like most about the system?
    \newline\_\_\_\_\_\_\_\_\_\_\_\_\_\_\_\_\_\_\_\_\_\_\_\_\_\_\_\_\_\_\_\_\_\_\_
    \item What difficulties did you encounter?
    \newline\_\_\_\_\_\_\_\_\_\_\_\_\_\_\_\_\_\_\_\_\_\_\_\_\_\_\_\_\_\_\_\_\_\_\_
    \item Any additional comments or suggestions?
    \newline\_\_\_\_\_\_\_\_\_\_\_\_\_\_\_\_\_\_\_\_\_\_\_\_\_\_\_\_\_\_\_\_\_\_\_
\end{itemize}

\textbf{Submission:}
Please submit your responses via email to \texttt{research@domain.com} or upload them to the project repository.\\

Thank you for your participation!


\end{document}
