\documentclass[12pt, titlepage]{article}

\usepackage{booktabs}
\usepackage{tabularx}
\usepackage{hyperref}
\hypersetup{
    colorlinks,
    citecolor=blue,
    filecolor=black,
    linkcolor=red,
    urlcolor=blue
}
\usepackage[round]{natbib}

%% Comments

\usepackage{color}

\newif\ifcomments\commentstrue %displays comments
%\newif\ifcomments\commentsfalse %so that comments do not display

\ifcomments
\newcommand{\authornote}[3]{\textcolor{#1}{[#3 ---#2]}}
\newcommand{\todo}[1]{\textcolor{red}{[TODO: #1]}}
\else
\newcommand{\authornote}[3]{}
\newcommand{\todo}[1]{}
\fi

\newcommand{\wss}[1]{\authornote{blue}{SS}{#1}} 
\newcommand{\plt}[1]{\authornote{magenta}{TPLT}{#1}} %For explanation of the template
\newcommand{\an}[1]{\authornote{cyan}{Author}{#1}}

%% Common Parts

\newcommand{\progname}{ProgName} % PUT YOUR PROGRAM NAME HERE
\newcommand{\authname}{Team \#, Team Name
\\ Student 1 name
\\ Student 2 name
\\ Student 3 name
\\ Student 4 name} % AUTHOR NAMES                  

\usepackage{hyperref}
    \hypersetup{colorlinks=true, linkcolor=blue, citecolor=blue, filecolor=blue,
                urlcolor=blue, unicode=false}
    \urlstyle{same}
                                


\begin{document}

\title{System Verification and Validation Plan for \progname{}}
\author{\authname}
\date{\today}
\maketitle

\upagenumbering{roman}

\section*{Revision History}

\begin{tabularx}{\textwidth}{p{3cm}p{2cm}X}
\toprule {\bf Date} & {\bf Version} & {\bf Notes}\\
\midrule
Date 1 & 1.0 & Notes\\
Date 2 & 1.1 & Notes\\
\bottomrule
\end{tabularx}

~\\
\wss{The intention of the VnV plan is to increase confidence in the software.
However, this does not mean listing every verification and validation technique
that has ever been devised.  The VnV plan should also be a \textbf{feasible}
plan. Execution of the plan should be possible with the time and team available.
If the full plan cannot be completed during the time available, it can either be
modified to ``fake it'', or a better solution is to add a section describing
what work has been completed and what work is still planned for the future.}

\wss{The VnV plan is typically started after the requirements stage, but before
the design stage.  This means that the sections related to unit testing cannot
initially be completed.  The sections will be filled in after the design stage
is complete.  the final version of the VnV plan should have all sections filled
in.}

\newpage

\tableofcontents

\listoftables
\wss{Remove this section if it isn't needed}

\listoffigures
\wss{Remove this section if it isn't needed}

\newpage

\section{Symbols, Abbreviations, and Acronyms}

\renewcommand{\arraystretch}{1.2}
\begin{tabular}{l l}
  \toprule
  \textbf{symbol} & \textbf{description}\\
  \midrule
  T & Test\\
  \bottomrule
\end{tabular}\\

\wss{symbols, abbreviations, or acronyms --- you can simply reference the SRS
  \citep{SRS} tables, if appropriate}

\wss{Remove this section if it isn't needed}

\newpage

\pagenumbering{arabic}

This document ... \wss{provide an introductory blurb and roadmap of the
  Verification and Validation plan}

\section{General Information}

\subsection{Summary}
This document reviews the verification and validation plan for Filtered Back
Projection (FBP) CT Image Reconstruction. The FBP method is used to reconstruct
cross-sectional images from projection data collected in CT scans. The accuracy
and performance of the reconstruction process are evaluated based on predefined
validation criteria. The system’s usability, maintainability, and reliability
are also assessed using user inputs and experimental results.

\subsection{Objectives}

\wss{State what is intended to be accomplished.  The objective will be around
  the qualities that are most important for your project.  You might have
  something like: ``build confidence in the software correctness,''
  ``demonstrate adequate usability.'' etc.  You won't list all of the qualities,
  just those that are most important.}

\wss{You should also list the objectives that are out of scope.  You don't have
the resources to do everything, so what will you be leaving out.  For instance,
if you are not going to verify the quality of usability, state this.  It is also
worthwhile to justify why the objectives are left out.}

\wss{The objectives are important because they highlight that you are aware of
limitations in your resources for verification and validation.  You can't do everything,
so what are you going to prioritize?  As an example, if your system depends on an
external library, you can explicitly state that you will assume that external library
has already been verified by its implementation team.}

\subsection{Challenge Level and Extras}

\wss{State the challenge level (advanced, general, basic) for your project.
Your challenge level should exactly match what is included in your problem
statement.  This should be the challenge level agreed on between you and the
course instructor.  You can use a pull request to update your challenge level
(in TeamComposition.csv or Repos.csv) if your plan changes as a result of the
VnV planning exercise.}

\wss{Summarize the extras (if any) that were tackled by this project.  Extras
can include usability testing, code walkthroughs, user documentation, formal
proof, GenderMag personas, Design Thinking, etc.  Extras should have already
been approved by the course instructor as included in your problem statement.
You can use a pull request to update your extras (in TeamComposition.csv or
Repos.csv) if your plan changes as a result of the VnV planning exercise.}

\subsection{Relevant Documentation}

\wss{Reference relevant documentation.  This will definitely include your SRS
  and your other project documents (design documents, like MG, MIS, etc).  You
  can include these even before they are written, since by the time the project
  is done, they will be written.  You can create BibTeX entries for your
  documents and within those entries include a hyperlink to the documents.}

\citet{SRS}

\wss{Don't just list the other documents.  You should explain why they are relevant and
how they relate to your VnV efforts.}

\section{Plan}
This section describes the verification and validation plan for the
\progname.The planning starts with the verification and validation team in
Section \ref{3.1}, followed by the SRS verification plan (Section \ref{3.2}),
design verification plan (Section \ref{3.3}), verification and validation
verification plan (Section \ref{3.4}), implementation verification plan (Section
\ref{3.5}), Automated testing and verification tools (Section \ref{3.6}), and
Software validation plan (Section \ref{3.7}).

\subsection{Verification and Validation Team} \label{3.1}
\begin{center}
\begin{table}[h]
\resizebox{\textwidth}{ %
    \begin{tabular}{ |l|l|p{2cm}|p{5cm}| }
    \hline
    \textbf{Name} & \textbf{Document} & \textbf{Role} & \textbf{Description} \\
    \hline
     Dr. Spencer Smith & All & Instructor/ Reviewer & Review the documents, design and documentation style. \\
     \hline
     Qianlin Chen & All & Author & Create and manage all documentation, develop the VnV plan, conduct VnV testing, and verify the implementation.\\
     \hline
     Xunzhou Ye & All & Domain Expert Reviewer & Review all the documents. \\
    \hline
    \end{tabular} %
}
\caption{Verification and validation team}
\label{vnvteam}
\end{table}
\end{center}

\subsection{SRS Verification Plan} \label{3.2}
The SRS document for \progname will be verified using a structured review
process. The verification plan consists of the following steps:
\begin{description}
\item[Initial Review] \hfill \\
  Assigned reviewers, instructor (Dr. Smith) and domain expert
  (Xunzhou). will conduct a manual review of the SRS document. In this step, a
  structured
  \href{https://github.com/smiths/capTemplate/blob/9251702fdcb9800c59f6ed3d11d91e2bd62fca6d/docs/Checklists/SRS-Checklist.pdf}{SRS
    Checklist} will be used to systematically evaluate key aspects such as
  requirement completeness, consistency, feasibility, and traceability.
\item[Issue Tracking] \hfill \\
  eviewers will provide feedback by documenting identified issues in an issue
  tracker (GitHub Issues). Each issue will be
  assigned to the document owner for later revision.
\item[Revision] \hfill \\
  The document owner will address reported issues by making necessary
  modifications to the SRS. Any unresolved or disputed concerns will be
  discussed among the reviewers.
\item[Final Documentation] \hfill \\
  After revisions, a final review will be conducted to confirm that all
  identified issues have been adequately addressed.
\end{description}


\subsection{Design Verification Plan} \label{3.3} The design documentation,
including the Module Guide (MG) and Module Interface Specification (MIS), will
be reviewed to ensure accuracy and completeness. The verification process will
involve a static analysis approach, where assigned reviewers will inspect the
documents to confirm that they align with system requirements and architectural
design principles. Reviewers will provide feedback through an issue-tracking
system, allowing the document owner to address concerns systematically. \\
A structured
\href{https://github.com/smiths/capTemplate/blob/9251702fdcb9800c59f6ed3d11d91e2bd62fca6d/docs/Checklists/MG-Checklist.pdf}{MG
  Checklist} and
\href{https://github.com/smiths/capTemplate/blob/9251702fdcb9800c59f6ed3d11d91e2bd62fca6d/docs/Checklists/MIS-Checklist.pdf}{MIS
  Checklist} will be used to ensure consistency, correctness, and traceability.
The finalized documents will be reviewed again to validate that all identified
issues have been resolved.

\subsection{Verification and Validation Plan Verification Plan} \label{3.4} The
review process for Verification and Validation (VnV) Plan will involve assigned
team members in Table \ref{vnvteam} conducting a structured inspection to verify that the plan aligns
with project requirements and verification strategies.
Identified issues will be documented in an issue-tracking system for resolution.\\
A
\href{https://github.com/smiths/capTemplate/blob/9251702fdcb9800c59f6ed3d11d91e2bd62fca6d/docs/Checklists/VnV-Checklist.pdf}{VnV
  Checklist} will be used to assess key aspects of the plan, including coverage
of testing methodologies and traceability of requirements. The final review will
confirm that all necessary corrections have been implemented before approval.

\subsection{Implementation Verification Plan} \label{3.5}

\wss{You should at least point to the tests listed in this document and the unit
  testing plan.}

\wss{In this section you would also give any details of any plans for static
  verification of the implementation.  Potential techniques include code
  walkthroughs, code inspection, static analyzers, etc.}

\wss{The final class presentation in CAS 741 could be used as a code
walkthrough.  There is also a possibility of using the final presentation (in
CAS741) for a partial usability survey.}

\subsection{Automated Testing and Verification Tools} \label{3.6} To ensure the
correctness and efficiency of the \progname, the following automated
testing and verification tools will be utilized:
\begin{description}
\item[Unit Testing] \hfill \\
  \href{https://docs.pytest.org/en/stable/}{pytest} will be used to test
  individual functions, such as backprojection and filtering operations, to
  verify that they produce the expected output given test input data.
\item[Performance Analysis] \hfill \\
  \href{https://docs.python.org/3/library/profile.html#module-cprofile}{cProfill}
  will analyze execution time and identify performance bottlenecks in
  computationally intensive functions like fourier transforms and interpolation.
\item[Static Analysis] \hfill \\
  \href{https://flake8.pycqa.org/en/latest/}{flake8} will enforce python code
  standards, ensuring maintainability and readability.
\item[Test Coverage] \hfill \\
  \href{https://coverage.readthedocs.io/en/7.6.12/}{coverage.py} will measure
  test coverage to ensure that all critical functions are tested and validated
  against various datasets.
\item[Visualization] \hfill \\
  \href{https://matplotlib.org/}{matplotlib} will be used to visualize sinograms
  and reconstructed images, allowing for manual verification of reconstruction
  accuracy.
\end{description}
Continuous Integration (CI) tools are unnecessary for this project since it does
not require automated builds, tests, or deployments like a web application.
Additionally, visual verification is crucial for image reconstruction, and CI
automation cannot replace the need for manual inspection of reconstructed
images.

\subsection{Software Validation Plan} \label{3.7}

\wss{If there is any external data that can be used for validation, you should
  point to it here.  If there are no plans for validation, you should state that
  here.}

\wss{You might want to use review sessions with the stakeholder to check that
the requirements document captures the right requirements.  Maybe task based
inspection?}

\wss{For those capstone teams with an external supervisor, the Rev 0 demo should
be used as an opportunity to validate the requirements.  You should plan on
demonstrating your project to your supervisor shortly after the scheduled Rev 0 demo.
The feedback from your supervisor will be very useful for improving your project.}

\wss{For teams without an external supervisor, user testing can serve the same purpose
as a Rev 0 demo for the supervisor.}

\wss{This section might reference back to the SRS verification section.}

\section{System Tests}

\wss{There should be text between all headings, even if it is just a roadmap of
the contents of the subsections.}

\subsection{Tests for Functional Requirements}
The functional requirements are divided into input and output
tests. \\
R1 correspond to input-related tests, ensuring that the system
correctly processes input images, projection data, and transformations.\\
R2 and R3 correspond to output-related tests, verifying the correctness of
Fourier filtering, back-projection, and reconstructed attenuation values.\\
While R3 operate on transformed projection data, it should be considered
output-related because they primarily contribute to intermediate and final
outputs of the reconstruction process, rather than defining new inputs.

\wss{Subsets of
  the tests may be in related, so this section is divided into different areas.
  If there are no identifiable subsets for the tests, this level of document
  structure can be removed.}

\wss{Include a blurb here to explain why the subsections below
  cover the requirements.  References to the SRS would be good here.}

\subsubsection{Input Tests}

\wss{It would be nice to have a blurb here to explain why the subsections below
  cover the requirements.  References to the SRS would be good here.  If a section
  covers tests for input constraints, you should reference the data constraints
  table in the SRS.}

\paragraph{Test for Input - Intensity Values}
\begin{enumerate}
\item{test-input-intensity-id1}
\begin{description}
\item[Control:] Automatic

\item[Initial State:] System is ready to accept a 2-D grayscale image for Radon Transform.

\item[Input:] A 2-D NumPy array (M by N matrix), where:
\begin{itemize}
    \item Number of row (M): Number of detector positions
    \item Number of column (N): Number of projection angles.
    \item Value in the matrix: X-ray intensity measurements, normalized between
      0 and 1.
\end{itemize}

\item[Output:] A valid 2-D matrix array (M by N matrix), where:
\begin{itemize}
    \item Number of row (M): Number of detector positions
    \item Number of column (N): Number of projection angles.
    \item Value in the matrix: Represent integrated projection data, between 0 and 1.
\end{itemize}

\item[Test Case Derivation:] This test ensures that the system correctly accepts a
valid grayscale 2D input image, applies the Radon Transform, and produces a
valid sinogram matrix. \\
The system should return an output that maintains consistency in shape and
correctly reflects the values after applying the log transform.

\item[How test will be performed:] The test will be automated using PyTest.
\end{description}
\end{enumerate}

\subsubsection{}

...

\subsection{Tests for Nonfunctional Requirements}

\wss{The nonfunctional requirements for accuracy will likely just reference the
  appropriate functional tests from above.  The test cases should mention
  reporting the relative error for these tests.  Not all projects will
  necessarily have nonfunctional requirements related to accuracy.}

\wss{For some nonfunctional tests, you won't be setting a target threshold for
passing the test, but rather describing the experiment you will do to measure
the quality for different inputs.  For instance, you could measure speed versus
the problem size.  The output of the test isn't pass/fail, but rather a summary
table or graph.}

\wss{Tests related to usability could include conducting a usability test and
  survey.  The survey will be in the Appendix.}

\wss{Static tests, review, inspections, and walkthroughs, will not follow the
format for the tests given below.}

\wss{If you introduce static tests in your plan, you need to provide details.
How will they be done?  In cases like code (or document) walkthroughs, who will
be involved? Be specific.}

\subsubsection{Area of Testing1}

\paragraph{Title for Test}

\begin{enumerate}

\item{test-id1\\}

Type: Functional, Dynamic, Manual, Static etc.

Initial State:

Input/Condition:

Output/Result:

How test will be performed:

\item{test-id2\\}

Type: Functional, Dynamic, Manual, Static etc.

Initial State:

Input:

Output:

How test will be performed:

\end{enumerate}

\subsubsection{Area of Testing2}

...

\subsection{Traceability Between Test Cases and Requirements}

\wss{Provide a table that shows which test cases are supporting which
  requirements.}

\section{Unit Test Description}

\wss{This section should not be filled in until after the MIS (detailed design
  document) has been completed.}

\wss{Reference your MIS (detailed design document) and explain your overall
philosophy for test case selection.}

\wss{To save space and time, it may be an option to provide less detail in this section.
For the unit tests you can potentially layout your testing strategy here.  That is, you
can explain how tests will be selected for each module.  For instance, your test building
approach could be test cases for each access program, including one test for normal behaviour
and as many tests as needed for edge cases.  Rather than create the details of the input
and output here, you could point to the unit testing code.  For this to work, you code
needs to be well-documented, with meaningful names for all of the tests.}

\subsection{Unit Testing Scope}

\wss{What modules are outside of the scope.  If there are modules that are
  developed by someone else, then you would say here if you aren't planning on
  verifying them.  There may also be modules that are part of your software, but
  have a lower priority for verification than others.  If this is the case,
  explain your rationale for the ranking of module importance.}

\subsection{Tests for Functional Requirements}

\wss{Most of the verification will be through automated unit testing.  If
  appropriate specific modules can be verified by a non-testing based
  technique.  That can also be documented in this section.}

\subsubsection{Module 1}

\wss{Include a blurb here to explain why the subsections below cover the module.
  References to the MIS would be good.  You will want tests from a black box
  perspective and from a white box perspective.  Explain to the reader how the
  tests were selected.}

\begin{enumerate}

\item{test-id1\\}

Type: \wss{Functional, Dynamic, Manual, Automatic, Static etc. Most will
  be automatic}

Initial State:

Input:

Output: \wss{The expected result for the given inputs}

Test Case Derivation: \wss{Justify the expected value given in the Output field}

How test will be performed:

\item{test-id2\\}

Type: \wss{Functional, Dynamic, Manual, Automatic, Static etc. Most will
  be automatic}

Initial State:

Input:

Output: \wss{The expected result for the given inputs}

Test Case Derivation: \wss{Justify the expected value given in the Output field}

How test will be performed:

\item{...\\}

\end{enumerate}

\subsubsection{Module 2}

...

\subsection{Tests for Nonfunctional Requirements}

\wss{If there is a module that needs to be independently assessed for
  performance, those test cases can go here.  In some projects, planning for
  nonfunctional tests of units will not be that relevant.}

\wss{These tests may involve collecting performance data from previously
  mentioned functional tests.}

\subsubsection{Module ?}

\begin{enumerate}

\item{test-id1\\}

Type: \wss{Functional, Dynamic, Manual, Automatic, Static etc. Most will
  be automatic}

Initial State:

Input/Condition:

Output/Result:

How test will be performed:

\item{test-id2\\}

Type: Functional, Dynamic, Manual, Static etc.

Initial State:

Input:

Output:

How test will be performed:

\end{enumerate}

\subsubsection{Module ?}

...

\subsection{Traceability Between Test Cases and Modules}

\wss{Provide evidence that all of the modules have been considered.}

\bibliographystyle{plainnat}

\bibliography{../../refs/References}

\newpage

\section{Appendix}

This is where you can place additional information.

\subsection{Symbolic Parameters}

The definition of the test cases will call for SYMBOLIC\_CONSTANTS.
Their values are defined in this section for easy maintenance.

\subsection{Usability Survey Questions?}

\wss{This is a section that would be appropriate for some projects.}

\newpage{}
\section*{Appendix --- Reflection}

\wss{This section is not required for CAS 741}

The information in this section will be used to evaluate the team members on the
graduate attribute of Lifelong Learning.

The purpose of reflection questions is to give you a chance to assess your own
learning and that of your group as a whole, and to find ways to improve in the
future. Reflection is an important part of the learning process.  Reflection is
also an essential component of a successful software development process.  

Reflections are most interesting and useful when they're honest, even if the
stories they tell are imperfect. You will be marked based on your depth of
thought and analysis, and not based on the content of the reflections
themselves. Thus, for full marks we encourage you to answer openly and honestly
and to avoid simply writing ``what you think the evaluator wants to hear.''

Please answer the following questions.  Some questions can be answered on the
team level, but where appropriate, each team member should write their own
response:


\begin{enumerate}
  \item What went well while writing this deliverable?
  \item What pain points did you experience during this deliverable, and how
    did you resolve them?
  \item What knowledge and skills will the team collectively need to acquire to
  successfully complete the verification and validation of your project?
  Examples of possible knowledge and skills include dynamic testing knowledge,
  static testing knowledge, specific tool usage, Valgrind etc.  You should look to
  identify at least one item for each team member.
  \item For each of the knowledge areas and skills identified in the previous
  question, what are at least two approaches to acquiring the knowledge or
  mastering the skill?  Of the identified approaches, which will each team
  member pursue, and why did they make this choice?
\end{enumerate}

\end{document}
